% Options for packages loaded elsewhere
\PassOptionsToPackage{unicode}{hyperref}
\PassOptionsToPackage{hyphens}{url}
\documentclass[
]{article}
\usepackage{xcolor}
\usepackage{amsmath,amssymb}
\setcounter{secnumdepth}{-\maxdimen} % remove section numbering
\usepackage{iftex}
\ifPDFTeX
  \usepackage[T1]{fontenc}
  \usepackage[utf8]{inputenc}
  \usepackage{textcomp} % provide euro and other symbols
\else % if luatex or xetex
  \usepackage{unicode-math} % this also loads fontspec
  \defaultfontfeatures{Scale=MatchLowercase}
  \defaultfontfeatures[\rmfamily]{Ligatures=TeX,Scale=1}
\fi
\usepackage{lmodern}
\ifPDFTeX\else
  % xetex/luatex font selection
\fi
% Use upquote if available, for straight quotes in verbatim environments
\IfFileExists{upquote.sty}{\usepackage{upquote}}{}
\IfFileExists{microtype.sty}{% use microtype if available
  \usepackage[]{microtype}
  \UseMicrotypeSet[protrusion]{basicmath} % disable protrusion for tt fonts
}{}
\makeatletter
\@ifundefined{KOMAClassName}{% if non-KOMA class
  \IfFileExists{parskip.sty}{%
    \usepackage{parskip}
  }{% else
    \setlength{\parindent}{0pt}
    \setlength{\parskip}{6pt plus 2pt minus 1pt}}
}{% if KOMA class
  \KOMAoptions{parskip=half}}
\makeatother
\setlength{\emergencystretch}{3em} % prevent overfull lines
\providecommand{\tightlist}{%
  \setlength{\itemsep}{0pt}\setlength{\parskip}{0pt}}
\usepackage{bookmark}
\IfFileExists{xurl.sty}{\usepackage{xurl}}{} % add URL line breaks if available
\urlstyle{same}
\hypersetup{
  pdftitle={K-Means},
  hidelinks,
  pdfcreator={LaTeX via pandoc}}

\title{K-Means}
\author{}
\date{}

\begin{document}
\maketitle

\subsection{Algorithm description}\label{algorithm-description}

\textbf{Approach:} Cluster data by separate samples to n groups of equal
variance.\\
The objective is to minimise inertia (Within cluster sum of squares
criterion).

\textbf{Inertia criterion equation:}

\textbf{Algorithm}

\begin{enumerate}
\tightlist
\item
  Choose centroids

  loop until number of iterations is met\\
  \strut \\
  \textbf{Loop body}:
\item
  Assign each sample to its nearest centroid (label update step)

  \hfill\break
\item
  Recompute centroids for the next iteration by taking mean of alll
  samples assigned to each centroid.\\
  \strut \\
  If new centroids - old centroids is smaller than a threshold, break
  out of the loop. If the inertia calculation from new centroids
  isn\textquotesingle t smaller than current minimum
  don\textquotesingle t reassign the centroids.
\end{enumerate}

\textbf{Scalability}: Scales well.

\textbf{Disadvantages}:

Clusters assumed to be convex and isotropic (Separable and of equal
variance).

That is to say, the algorithm will not perform well on clusters with
irregular shapes/elongated clusters

Inertia is not a normalised metric. In very high-dimensional spaces, can
suffer from the curse of dimensionality

\subsection{Coded algorithm with sample
dataset}\label{coded-algorithm-with-sample-dataset}

\end{document}
